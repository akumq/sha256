\documentclass[11pt]{article}

    \usepackage[breakable]{tcolorbox}
    \usepackage{parskip} % Stop auto-indenting (to mimic markdown behaviour)
    

    % Basic figure setup, for now with no caption control since it's done
    % automatically by Pandoc (which extracts ![](path) syntax from Markdown).
    \usepackage{graphicx}
    % Keep aspect ratio if custom image width or height is specified
    \setkeys{Gin}{keepaspectratio}
    % Maintain compatibility with old templates. Remove in nbconvert 6.0
    \let\Oldincludegraphics\includegraphics
    % Ensure that by default, figures have no caption (until we provide a
    % proper Figure object with a Caption API and a way to capture that
    % in the conversion process - todo).
    \usepackage{caption}
    \DeclareCaptionFormat{nocaption}{}
    \captionsetup{format=nocaption,aboveskip=0pt,belowskip=0pt}

    \usepackage{float}
    \floatplacement{figure}{H} % forces figures to be placed at the correct location
    \usepackage{xcolor} % Allow colors to be defined
    \usepackage{enumerate} % Needed for markdown enumerations to work
    \usepackage{geometry} % Used to adjust the document margins
    \usepackage{amsmath} % Equations
    \usepackage{amssymb} % Equations
    \usepackage{textcomp} % defines textquotesingle
    % Hack from http://tex.stackexchange.com/a/47451/13684:
    \AtBeginDocument{%
        \def\PYZsq{\textquotesingle}% Upright quotes in Pygmentized code
    }
    \usepackage{upquote} % Upright quotes for verbatim code
    \usepackage{eurosym} % defines \euro

    \usepackage{iftex}
    \ifPDFTeX
        \usepackage[T1]{fontenc}
        \IfFileExists{alphabeta.sty}{
              \usepackage{alphabeta}
          }{
              \usepackage[mathletters]{ucs}
              \usepackage[utf8x]{inputenc}
          }
    \else
        \usepackage{fontspec}
        \usepackage{unicode-math}
    \fi

    \usepackage{fancyvrb} % verbatim replacement that allows latex
    \usepackage{grffile} % extends the file name processing of package graphics
                         % to support a larger range
    \makeatletter % fix for old versions of grffile with XeLaTeX
    \@ifpackagelater{grffile}{2019/11/01}
    {
      % Do nothing on new versions
    }
    {
      \def\Gread@@xetex#1{%
        \IfFileExists{"\Gin@base".bb}%
        {\Gread@eps{\Gin@base.bb}}%
        {\Gread@@xetex@aux#1}%
      }
    }
    \makeatother
    \usepackage[Export]{adjustbox} % Used to constrain images to a maximum size
    \adjustboxset{max size={0.9\linewidth}{0.9\paperheight}}

    % The hyperref package gives us a pdf with properly built
    % internal navigation ('pdf bookmarks' for the table of contents,
    % internal cross-reference links, web links for URLs, etc.)
    \usepackage{hyperref}
    % The default LaTeX title has an obnoxious amount of whitespace. By default,
    % titling removes some of it. It also provides customization options.
    \usepackage{titling}
    \usepackage{longtable} % longtable support required by pandoc >1.10
    \usepackage{booktabs}  % table support for pandoc > 1.12.2
    \usepackage{array}     % table support for pandoc >= 2.11.3
    \usepackage{calc}      % table minipage width calculation for pandoc >= 2.11.1
    \usepackage[inline]{enumitem} % IRkernel/repr support (it uses the enumerate* environment)
    \usepackage[normalem]{ulem} % ulem is needed to support strikethroughs (\sout)
                                % normalem makes italics be italics, not underlines
    \usepackage{soul}      % strikethrough (\st) support for pandoc >= 3.0.0
    \usepackage{mathrsfs}
    

    
    % Colors for the hyperref package
    \definecolor{urlcolor}{rgb}{0,.145,.698}
    \definecolor{linkcolor}{rgb}{.71,0.21,0.01}
    \definecolor{citecolor}{rgb}{.12,.54,.11}

    % ANSI colors
    \definecolor{ansi-black}{HTML}{3E424D}
    \definecolor{ansi-black-intense}{HTML}{282C36}
    \definecolor{ansi-red}{HTML}{E75C58}
    \definecolor{ansi-red-intense}{HTML}{B22B31}
    \definecolor{ansi-green}{HTML}{00A250}
    \definecolor{ansi-green-intense}{HTML}{007427}
    \definecolor{ansi-yellow}{HTML}{DDB62B}
    \definecolor{ansi-yellow-intense}{HTML}{B27D12}
    \definecolor{ansi-blue}{HTML}{208FFB}
    \definecolor{ansi-blue-intense}{HTML}{0065CA}
    \definecolor{ansi-magenta}{HTML}{D160C4}
    \definecolor{ansi-magenta-intense}{HTML}{A03196}
    \definecolor{ansi-cyan}{HTML}{60C6C8}
    \definecolor{ansi-cyan-intense}{HTML}{258F8F}
    \definecolor{ansi-white}{HTML}{C5C1B4}
    \definecolor{ansi-white-intense}{HTML}{A1A6B2}
    \definecolor{ansi-default-inverse-fg}{HTML}{FFFFFF}
    \definecolor{ansi-default-inverse-bg}{HTML}{000000}

    % common color for the border for error outputs.
    \definecolor{outerrorbackground}{HTML}{FFDFDF}

    % commands and environments needed by pandoc snippets
    % extracted from the output of `pandoc -s`
    \providecommand{\tightlist}{%
      \setlength{\itemsep}{0pt}\setlength{\parskip}{0pt}}
    \DefineVerbatimEnvironment{Highlighting}{Verbatim}{commandchars=\\\{\}}
    % Add ',fontsize=\small' for more characters per line
    \newenvironment{Shaded}{}{}
    \newcommand{\KeywordTok}[1]{\textcolor[rgb]{0.00,0.44,0.13}{\textbf{{#1}}}}
    \newcommand{\DataTypeTok}[1]{\textcolor[rgb]{0.56,0.13,0.00}{{#1}}}
    \newcommand{\DecValTok}[1]{\textcolor[rgb]{0.25,0.63,0.44}{{#1}}}
    \newcommand{\BaseNTok}[1]{\textcolor[rgb]{0.25,0.63,0.44}{{#1}}}
    \newcommand{\FloatTok}[1]{\textcolor[rgb]{0.25,0.63,0.44}{{#1}}}
    \newcommand{\CharTok}[1]{\textcolor[rgb]{0.25,0.44,0.63}{{#1}}}
    \newcommand{\StringTok}[1]{\textcolor[rgb]{0.25,0.44,0.63}{{#1}}}
    \newcommand{\CommentTok}[1]{\textcolor[rgb]{0.38,0.63,0.69}{\textit{{#1}}}}
    \newcommand{\OtherTok}[1]{\textcolor[rgb]{0.00,0.44,0.13}{{#1}}}
    \newcommand{\AlertTok}[1]{\textcolor[rgb]{1.00,0.00,0.00}{\textbf{{#1}}}}
    \newcommand{\FunctionTok}[1]{\textcolor[rgb]{0.02,0.16,0.49}{{#1}}}
    \newcommand{\RegionMarkerTok}[1]{{#1}}
    \newcommand{\ErrorTok}[1]{\textcolor[rgb]{1.00,0.00,0.00}{\textbf{{#1}}}}
    \newcommand{\NormalTok}[1]{{#1}}

    % Additional commands for more recent versions of Pandoc
    \newcommand{\ConstantTok}[1]{\textcolor[rgb]{0.53,0.00,0.00}{{#1}}}
    \newcommand{\SpecialCharTok}[1]{\textcolor[rgb]{0.25,0.44,0.63}{{#1}}}
    \newcommand{\VerbatimStringTok}[1]{\textcolor[rgb]{0.25,0.44,0.63}{{#1}}}
    \newcommand{\SpecialStringTok}[1]{\textcolor[rgb]{0.73,0.40,0.53}{{#1}}}
    \newcommand{\ImportTok}[1]{{#1}}
    \newcommand{\DocumentationTok}[1]{\textcolor[rgb]{0.73,0.13,0.13}{\textit{{#1}}}}
    \newcommand{\AnnotationTok}[1]{\textcolor[rgb]{0.38,0.63,0.69}{\textbf{\textit{{#1}}}}}
    \newcommand{\CommentVarTok}[1]{\textcolor[rgb]{0.38,0.63,0.69}{\textbf{\textit{{#1}}}}}
    \newcommand{\VariableTok}[1]{\textcolor[rgb]{0.10,0.09,0.49}{{#1}}}
    \newcommand{\ControlFlowTok}[1]{\textcolor[rgb]{0.00,0.44,0.13}{\textbf{{#1}}}}
    \newcommand{\OperatorTok}[1]{\textcolor[rgb]{0.40,0.40,0.40}{{#1}}}
    \newcommand{\BuiltInTok}[1]{{#1}}
    \newcommand{\ExtensionTok}[1]{{#1}}
    \newcommand{\PreprocessorTok}[1]{\textcolor[rgb]{0.74,0.48,0.00}{{#1}}}
    \newcommand{\AttributeTok}[1]{\textcolor[rgb]{0.49,0.56,0.16}{{#1}}}
    \newcommand{\InformationTok}[1]{\textcolor[rgb]{0.38,0.63,0.69}{\textbf{\textit{{#1}}}}}
    \newcommand{\WarningTok}[1]{\textcolor[rgb]{0.38,0.63,0.69}{\textbf{\textit{{#1}}}}}
    \makeatletter
    \newsavebox\pandoc@box
    \newcommand*\pandocbounded[1]{%
      \sbox\pandoc@box{#1}%
      % scaling factors for width and height
      \Gscale@div\@tempa\textheight{\dimexpr\ht\pandoc@box+\dp\pandoc@box\relax}%
      \Gscale@div\@tempb\linewidth{\wd\pandoc@box}%
      % select the smaller of both
      \ifdim\@tempb\p@<\@tempa\p@
        \let\@tempa\@tempb
      \fi
      % scaling accordingly (\@tempa < 1)
      \ifdim\@tempa\p@<\p@
        \scalebox{\@tempa}{\usebox\pandoc@box}%
      % scaling not needed, use as it is
      \else
        \usebox{\pandoc@box}%
      \fi
    }
    \makeatother

    % Define a nice break command that doesn't care if a line doesn't already
    % exist.
    \def\br{\hspace*{\fill} \\* }
    % Math Jax compatibility definitions
    \def\gt{>}
    \def\lt{<}
    \let\Oldtex\TeX
    \let\Oldlatex\LaTeX
    \renewcommand{\TeX}{\textrm{\Oldtex}}
    \renewcommand{\LaTeX}{\textrm{\Oldlatex}}
    % Document parameters
    % Document title
    \title{sha256}
    
    
    
    
    
    
    
% Pygments definitions
\makeatletter
\def\PY@reset{\let\PY@it=\relax \let\PY@bf=\relax%
    \let\PY@ul=\relax \let\PY@tc=\relax%
    \let\PY@bc=\relax \let\PY@ff=\relax}
\def\PY@tok#1{\csname PY@tok@#1\endcsname}
\def\PY@toks#1+{\ifx\relax#1\empty\else%
    \PY@tok{#1}\expandafter\PY@toks\fi}
\def\PY@do#1{\PY@bc{\PY@tc{\PY@ul{%
    \PY@it{\PY@bf{\PY@ff{#1}}}}}}}
\def\PY#1#2{\PY@reset\PY@toks#1+\relax+\PY@do{#2}}

\@namedef{PY@tok@w}{\def\PY@tc##1{\textcolor[rgb]{0.73,0.73,0.73}{##1}}}
\@namedef{PY@tok@c}{\let\PY@it=\textit\def\PY@tc##1{\textcolor[rgb]{0.24,0.48,0.48}{##1}}}
\@namedef{PY@tok@cp}{\def\PY@tc##1{\textcolor[rgb]{0.61,0.40,0.00}{##1}}}
\@namedef{PY@tok@k}{\let\PY@bf=\textbf\def\PY@tc##1{\textcolor[rgb]{0.00,0.50,0.00}{##1}}}
\@namedef{PY@tok@kp}{\def\PY@tc##1{\textcolor[rgb]{0.00,0.50,0.00}{##1}}}
\@namedef{PY@tok@kt}{\def\PY@tc##1{\textcolor[rgb]{0.69,0.00,0.25}{##1}}}
\@namedef{PY@tok@o}{\def\PY@tc##1{\textcolor[rgb]{0.40,0.40,0.40}{##1}}}
\@namedef{PY@tok@ow}{\let\PY@bf=\textbf\def\PY@tc##1{\textcolor[rgb]{0.67,0.13,1.00}{##1}}}
\@namedef{PY@tok@nb}{\def\PY@tc##1{\textcolor[rgb]{0.00,0.50,0.00}{##1}}}
\@namedef{PY@tok@nf}{\def\PY@tc##1{\textcolor[rgb]{0.00,0.00,1.00}{##1}}}
\@namedef{PY@tok@nc}{\let\PY@bf=\textbf\def\PY@tc##1{\textcolor[rgb]{0.00,0.00,1.00}{##1}}}
\@namedef{PY@tok@nn}{\let\PY@bf=\textbf\def\PY@tc##1{\textcolor[rgb]{0.00,0.00,1.00}{##1}}}
\@namedef{PY@tok@ne}{\let\PY@bf=\textbf\def\PY@tc##1{\textcolor[rgb]{0.80,0.25,0.22}{##1}}}
\@namedef{PY@tok@nv}{\def\PY@tc##1{\textcolor[rgb]{0.10,0.09,0.49}{##1}}}
\@namedef{PY@tok@no}{\def\PY@tc##1{\textcolor[rgb]{0.53,0.00,0.00}{##1}}}
\@namedef{PY@tok@nl}{\def\PY@tc##1{\textcolor[rgb]{0.46,0.46,0.00}{##1}}}
\@namedef{PY@tok@ni}{\let\PY@bf=\textbf\def\PY@tc##1{\textcolor[rgb]{0.44,0.44,0.44}{##1}}}
\@namedef{PY@tok@na}{\def\PY@tc##1{\textcolor[rgb]{0.41,0.47,0.13}{##1}}}
\@namedef{PY@tok@nt}{\let\PY@bf=\textbf\def\PY@tc##1{\textcolor[rgb]{0.00,0.50,0.00}{##1}}}
\@namedef{PY@tok@nd}{\def\PY@tc##1{\textcolor[rgb]{0.67,0.13,1.00}{##1}}}
\@namedef{PY@tok@s}{\def\PY@tc##1{\textcolor[rgb]{0.73,0.13,0.13}{##1}}}
\@namedef{PY@tok@sd}{\let\PY@it=\textit\def\PY@tc##1{\textcolor[rgb]{0.73,0.13,0.13}{##1}}}
\@namedef{PY@tok@si}{\let\PY@bf=\textbf\def\PY@tc##1{\textcolor[rgb]{0.64,0.35,0.47}{##1}}}
\@namedef{PY@tok@se}{\let\PY@bf=\textbf\def\PY@tc##1{\textcolor[rgb]{0.67,0.36,0.12}{##1}}}
\@namedef{PY@tok@sr}{\def\PY@tc##1{\textcolor[rgb]{0.64,0.35,0.47}{##1}}}
\@namedef{PY@tok@ss}{\def\PY@tc##1{\textcolor[rgb]{0.10,0.09,0.49}{##1}}}
\@namedef{PY@tok@sx}{\def\PY@tc##1{\textcolor[rgb]{0.00,0.50,0.00}{##1}}}
\@namedef{PY@tok@m}{\def\PY@tc##1{\textcolor[rgb]{0.40,0.40,0.40}{##1}}}
\@namedef{PY@tok@gh}{\let\PY@bf=\textbf\def\PY@tc##1{\textcolor[rgb]{0.00,0.00,0.50}{##1}}}
\@namedef{PY@tok@gu}{\let\PY@bf=\textbf\def\PY@tc##1{\textcolor[rgb]{0.50,0.00,0.50}{##1}}}
\@namedef{PY@tok@gd}{\def\PY@tc##1{\textcolor[rgb]{0.63,0.00,0.00}{##1}}}
\@namedef{PY@tok@gi}{\def\PY@tc##1{\textcolor[rgb]{0.00,0.52,0.00}{##1}}}
\@namedef{PY@tok@gr}{\def\PY@tc##1{\textcolor[rgb]{0.89,0.00,0.00}{##1}}}
\@namedef{PY@tok@ge}{\let\PY@it=\textit}
\@namedef{PY@tok@gs}{\let\PY@bf=\textbf}
\@namedef{PY@tok@ges}{\let\PY@bf=\textbf\let\PY@it=\textit}
\@namedef{PY@tok@gp}{\let\PY@bf=\textbf\def\PY@tc##1{\textcolor[rgb]{0.00,0.00,0.50}{##1}}}
\@namedef{PY@tok@go}{\def\PY@tc##1{\textcolor[rgb]{0.44,0.44,0.44}{##1}}}
\@namedef{PY@tok@gt}{\def\PY@tc##1{\textcolor[rgb]{0.00,0.27,0.87}{##1}}}
\@namedef{PY@tok@err}{\def\PY@bc##1{{\setlength{\fboxsep}{\string -\fboxrule}\fcolorbox[rgb]{1.00,0.00,0.00}{1,1,1}{\strut ##1}}}}
\@namedef{PY@tok@kc}{\let\PY@bf=\textbf\def\PY@tc##1{\textcolor[rgb]{0.00,0.50,0.00}{##1}}}
\@namedef{PY@tok@kd}{\let\PY@bf=\textbf\def\PY@tc##1{\textcolor[rgb]{0.00,0.50,0.00}{##1}}}
\@namedef{PY@tok@kn}{\let\PY@bf=\textbf\def\PY@tc##1{\textcolor[rgb]{0.00,0.50,0.00}{##1}}}
\@namedef{PY@tok@kr}{\let\PY@bf=\textbf\def\PY@tc##1{\textcolor[rgb]{0.00,0.50,0.00}{##1}}}
\@namedef{PY@tok@bp}{\def\PY@tc##1{\textcolor[rgb]{0.00,0.50,0.00}{##1}}}
\@namedef{PY@tok@fm}{\def\PY@tc##1{\textcolor[rgb]{0.00,0.00,1.00}{##1}}}
\@namedef{PY@tok@vc}{\def\PY@tc##1{\textcolor[rgb]{0.10,0.09,0.49}{##1}}}
\@namedef{PY@tok@vg}{\def\PY@tc##1{\textcolor[rgb]{0.10,0.09,0.49}{##1}}}
\@namedef{PY@tok@vi}{\def\PY@tc##1{\textcolor[rgb]{0.10,0.09,0.49}{##1}}}
\@namedef{PY@tok@vm}{\def\PY@tc##1{\textcolor[rgb]{0.10,0.09,0.49}{##1}}}
\@namedef{PY@tok@sa}{\def\PY@tc##1{\textcolor[rgb]{0.73,0.13,0.13}{##1}}}
\@namedef{PY@tok@sb}{\def\PY@tc##1{\textcolor[rgb]{0.73,0.13,0.13}{##1}}}
\@namedef{PY@tok@sc}{\def\PY@tc##1{\textcolor[rgb]{0.73,0.13,0.13}{##1}}}
\@namedef{PY@tok@dl}{\def\PY@tc##1{\textcolor[rgb]{0.73,0.13,0.13}{##1}}}
\@namedef{PY@tok@s2}{\def\PY@tc##1{\textcolor[rgb]{0.73,0.13,0.13}{##1}}}
\@namedef{PY@tok@sh}{\def\PY@tc##1{\textcolor[rgb]{0.73,0.13,0.13}{##1}}}
\@namedef{PY@tok@s1}{\def\PY@tc##1{\textcolor[rgb]{0.73,0.13,0.13}{##1}}}
\@namedef{PY@tok@mb}{\def\PY@tc##1{\textcolor[rgb]{0.40,0.40,0.40}{##1}}}
\@namedef{PY@tok@mf}{\def\PY@tc##1{\textcolor[rgb]{0.40,0.40,0.40}{##1}}}
\@namedef{PY@tok@mh}{\def\PY@tc##1{\textcolor[rgb]{0.40,0.40,0.40}{##1}}}
\@namedef{PY@tok@mi}{\def\PY@tc##1{\textcolor[rgb]{0.40,0.40,0.40}{##1}}}
\@namedef{PY@tok@il}{\def\PY@tc##1{\textcolor[rgb]{0.40,0.40,0.40}{##1}}}
\@namedef{PY@tok@mo}{\def\PY@tc##1{\textcolor[rgb]{0.40,0.40,0.40}{##1}}}
\@namedef{PY@tok@ch}{\let\PY@it=\textit\def\PY@tc##1{\textcolor[rgb]{0.24,0.48,0.48}{##1}}}
\@namedef{PY@tok@cm}{\let\PY@it=\textit\def\PY@tc##1{\textcolor[rgb]{0.24,0.48,0.48}{##1}}}
\@namedef{PY@tok@cpf}{\let\PY@it=\textit\def\PY@tc##1{\textcolor[rgb]{0.24,0.48,0.48}{##1}}}
\@namedef{PY@tok@c1}{\let\PY@it=\textit\def\PY@tc##1{\textcolor[rgb]{0.24,0.48,0.48}{##1}}}
\@namedef{PY@tok@cs}{\let\PY@it=\textit\def\PY@tc##1{\textcolor[rgb]{0.24,0.48,0.48}{##1}}}

\def\PYZbs{\char`\\}
\def\PYZus{\char`\_}
\def\PYZob{\char`\{}
\def\PYZcb{\char`\}}
\def\PYZca{\char`\^}
\def\PYZam{\char`\&}
\def\PYZlt{\char`\<}
\def\PYZgt{\char`\>}
\def\PYZsh{\char`\#}
\def\PYZpc{\char`\%}
\def\PYZdl{\char`\$}
\def\PYZhy{\char`\-}
\def\PYZsq{\char`\'}
\def\PYZdq{\char`\"}
\def\PYZti{\char`\~}
% for compatibility with earlier versions
\def\PYZat{@}
\def\PYZlb{[}
\def\PYZrb{]}
\makeatother


    % For linebreaks inside Verbatim environment from package fancyvrb.
    \makeatletter
        \newbox\Wrappedcontinuationbox
        \newbox\Wrappedvisiblespacebox
        \newcommand*\Wrappedvisiblespace {\textcolor{red}{\textvisiblespace}}
        \newcommand*\Wrappedcontinuationsymbol {\textcolor{red}{\llap{\tiny$\m@th\hookrightarrow$}}}
        \newcommand*\Wrappedcontinuationindent {3ex }
        \newcommand*\Wrappedafterbreak {\kern\Wrappedcontinuationindent\copy\Wrappedcontinuationbox}
        % Take advantage of the already applied Pygments mark-up to insert
        % potential linebreaks for TeX processing.
        %        {, <, #, %, $, ' and ": go to next line.
        %        _, }, ^, &, >, - and ~: stay at end of broken line.
        % Use of \textquotesingle for straight quote.
        \newcommand*\Wrappedbreaksatspecials {%
            \def\PYGZus{\discretionary{\char`\_}{\Wrappedafterbreak}{\char`\_}}%
            \def\PYGZob{\discretionary{}{\Wrappedafterbreak\char`\{}{\char`\{}}%
            \def\PYGZcb{\discretionary{\char`\}}{\Wrappedafterbreak}{\char`\}}}%
            \def\PYGZca{\discretionary{\char`\^}{\Wrappedafterbreak}{\char`\^}}%
            \def\PYGZam{\discretionary{\char`\&}{\Wrappedafterbreak}{\char`\&}}%
            \def\PYGZlt{\discretionary{}{\Wrappedafterbreak\char`\<}{\char`\<}}%
            \def\PYGZgt{\discretionary{\char`\>}{\Wrappedafterbreak}{\char`\>}}%
            \def\PYGZsh{\discretionary{}{\Wrappedafterbreak\char`\#}{\char`\#}}%
            \def\PYGZpc{\discretionary{}{\Wrappedafterbreak\char`\%}{\char`\%}}%
            \def\PYGZdl{\discretionary{}{\Wrappedafterbreak\char`\$}{\char`\$}}%
            \def\PYGZhy{\discretionary{\char`\-}{\Wrappedafterbreak}{\char`\-}}%
            \def\PYGZsq{\discretionary{}{\Wrappedafterbreak\textquotesingle}{\textquotesingle}}%
            \def\PYGZdq{\discretionary{}{\Wrappedafterbreak\char`\"}{\char`\"}}%
            \def\PYGZti{\discretionary{\char`\~}{\Wrappedafterbreak}{\char`\~}}%
        }
        % Some characters . , ; ? ! / are not pygmentized.
        % This macro makes them "active" and they will insert potential linebreaks
        \newcommand*\Wrappedbreaksatpunct {%
            \lccode`\~`\.\lowercase{\def~}{\discretionary{\hbox{\char`\.}}{\Wrappedafterbreak}{\hbox{\char`\.}}}%
            \lccode`\~`\,\lowercase{\def~}{\discretionary{\hbox{\char`\,}}{\Wrappedafterbreak}{\hbox{\char`\,}}}%
            \lccode`\~`\;\lowercase{\def~}{\discretionary{\hbox{\char`\;}}{\Wrappedafterbreak}{\hbox{\char`\;}}}%
            \lccode`\~`\:\lowercase{\def~}{\discretionary{\hbox{\char`\:}}{\Wrappedafterbreak}{\hbox{\char`\:}}}%
            \lccode`\~`\?\lowercase{\def~}{\discretionary{\hbox{\char`\?}}{\Wrappedafterbreak}{\hbox{\char`\?}}}%
            \lccode`\~`\!\lowercase{\def~}{\discretionary{\hbox{\char`\!}}{\Wrappedafterbreak}{\hbox{\char`\!}}}%
            \lccode`\~`\/\lowercase{\def~}{\discretionary{\hbox{\char`\/}}{\Wrappedafterbreak}{\hbox{\char`\/}}}%
            \catcode`\.\active
            \catcode`\,\active
            \catcode`\;\active
            \catcode`\:\active
            \catcode`\?\active
            \catcode`\!\active
            \catcode`\/\active
            \lccode`\~`\~
        }
    \makeatother

    \let\OriginalVerbatim=\Verbatim
    \makeatletter
    \renewcommand{\Verbatim}[1][1]{%
        %\parskip\z@skip
        \sbox\Wrappedcontinuationbox {\Wrappedcontinuationsymbol}%
        \sbox\Wrappedvisiblespacebox {\FV@SetupFont\Wrappedvisiblespace}%
        \def\FancyVerbFormatLine ##1{\hsize\linewidth
            \vtop{\raggedright\hyphenpenalty\z@\exhyphenpenalty\z@
                \doublehyphendemerits\z@\finalhyphendemerits\z@
                \strut ##1\strut}%
        }%
        % If the linebreak is at a space, the latter will be displayed as visible
        % space at end of first line, and a continuation symbol starts next line.
        % Stretch/shrink are however usually zero for typewriter font.
        \def\FV@Space {%
            \nobreak\hskip\z@ plus\fontdimen3\font minus\fontdimen4\font
            \discretionary{\copy\Wrappedvisiblespacebox}{\Wrappedafterbreak}
            {\kern\fontdimen2\font}%
        }%

        % Allow breaks at special characters using \PYG... macros.
        \Wrappedbreaksatspecials
        % Breaks at punctuation characters . , ; ? ! and / need catcode=\active
        \OriginalVerbatim[#1,codes*=\Wrappedbreaksatpunct]%
    }
    \makeatother

    % Exact colors from NB
    \definecolor{incolor}{HTML}{303F9F}
    \definecolor{outcolor}{HTML}{D84315}
    \definecolor{cellborder}{HTML}{CFCFCF}
    \definecolor{cellbackground}{HTML}{F7F7F7}

    % prompt
    \makeatletter
    \newcommand{\boxspacing}{\kern\kvtcb@left@rule\kern\kvtcb@boxsep}
    \makeatother
    \newcommand{\prompt}[4]{
        {\ttfamily\llap{{\color{#2}[#3]:\hspace{3pt}#4}}\vspace{-\baselineskip}}
    }
    

    
    % Prevent overflowing lines due to hard-to-break entities
    \sloppy
    % Setup hyperref package
    \hypersetup{
      breaklinks=true,  % so long urls are correctly broken across lines
      colorlinks=true,
      urlcolor=urlcolor,
      linkcolor=linkcolor,
      citecolor=citecolor,
      }
    % Slightly bigger margins than the latex defaults
    
    \geometry{verbose,tmargin=1in,bmargin=1in,lmargin=1in,rmargin=1in}
    
    

\begin{document}
    
    \maketitle
    
    

    
    \subsection{SHA-256}\label{sha-256}

L'algorithme SHA-256, conçu par la NSA est un algorithme de hashage qui
permet de hasher n'importe qu'elle type de donnée, l'algorithme sha-256
fourniras toujours une donnée de 256 bits, peu importe la taille de la
donnée fournis, qu'elle soit supérieur ou inférieur a 256, une valeur
retourneras toujours le même Hash, néanmoins si elle change même d'un
seul caractère elle ce retrouveras alors complétement différentes.

pour présenter l'algorithme nous allons le faire étape par étape par une
démonstration guidé:

    \subsection{Démonstration guidée (version
simple)}\label{duxe9monstration-guiduxe9e-version-simple}

Pour présenter l'algorithme, on suit un même message pas à pas :

\begin{enumerate}
\def\labelenumi{\arabic{enumi})}
\tightlist
\item
  Point de départ

  \begin{itemize}
  \tightlist
  \item
    On part de 8 nombres fixes (valeurs initiales) qui servent d'état de
    travail. On en fait une copie pour commencer le calcul.
  \end{itemize}
\item
  Préparation du message

  \begin{itemize}
  \tightlist
  \item
    On transforme le message en bits.
  \item
    On ajoute un bit ``1'', puis des ``0'', puis la taille du message
    sur 64 bits, pour que la longueur totale soit un multiple de 512.
  \end{itemize}
\item
  Traitement par blocs de 512 bits

  \begin{itemize}
  \tightlist
  \item
    On découpe chaque bloc en 16 ``mots'' de 32 bits, puis on en
    fabrique 48 de plus pour obtenir 64 mots.
  \item
    On fait 64 ``tours'' de mélange de l'état avec ces mots et des
    constantes. C'est là que l'information du message se diffuse et se
    brouille fortement.
  \end{itemize}
\item
  Résultat final

  \begin{itemize}
  \tightlist
  \item
    Après tous les blocs, on obtient 8 nombres finaux. On les écrit en
    hexadécimal l'un à la suite de l'autre pour former une empreinte (64
    caractères) : c'est le SHA‑256 du message.
  \end{itemize}
\end{enumerate}

    \begin{tcolorbox}[breakable, size=fbox, boxrule=1pt, pad at break*=1mm,colback=cellbackground, colframe=cellborder]
\prompt{In}{incolor}{60}{\boxspacing}
\begin{Verbatim}[commandchars=\\\{\}]
\PY{n}{K} \PY{o}{=} \PY{p}{[}
    \PY{l+m+mh}{0x428A2F98}\PY{p}{,} \PY{l+m+mh}{0x71374491}\PY{p}{,} \PY{l+m+mh}{0xB5C0FBCF}\PY{p}{,} \PY{l+m+mh}{0xE9B5DBA5}\PY{p}{,} \PY{l+m+mh}{0x3956C25B}\PY{p}{,} \PY{l+m+mh}{0x59F111F1}\PY{p}{,}
    \PY{l+m+mh}{0x923F82A4}\PY{p}{,} \PY{l+m+mh}{0xAB1C5ED5}\PY{p}{,} \PY{l+m+mh}{0xD807AA98}\PY{p}{,} \PY{l+m+mh}{0x12835B01}\PY{p}{,} \PY{l+m+mh}{0x243185BE}\PY{p}{,} \PY{l+m+mh}{0x550C7DC3}\PY{p}{,}
    \PY{l+m+mh}{0x72BE5D74}\PY{p}{,} \PY{l+m+mh}{0x80DEB1FE}\PY{p}{,} \PY{l+m+mh}{0x9BDC06A7}\PY{p}{,} \PY{l+m+mh}{0xC19BF174}\PY{p}{,} \PY{l+m+mh}{0xE49B69C1}\PY{p}{,} \PY{l+m+mh}{0xEFBE4786}\PY{p}{,}
    \PY{l+m+mh}{0x0FC19DC6}\PY{p}{,} \PY{l+m+mh}{0x240CA1CC}\PY{p}{,} \PY{l+m+mh}{0x2DE92C6F}\PY{p}{,} \PY{l+m+mh}{0x4A7484AA}\PY{p}{,} \PY{l+m+mh}{0x5CB0A9DC}\PY{p}{,} \PY{l+m+mh}{0x76F988DA}\PY{p}{,}
    \PY{l+m+mh}{0x983E5152}\PY{p}{,} \PY{l+m+mh}{0xA831C66D}\PY{p}{,} \PY{l+m+mh}{0xB00327C8}\PY{p}{,} \PY{l+m+mh}{0xBF597FC7}\PY{p}{,} \PY{l+m+mh}{0xC6E00BF3}\PY{p}{,} \PY{l+m+mh}{0xD5A79147}\PY{p}{,}
    \PY{l+m+mh}{0x06CA6351}\PY{p}{,} \PY{l+m+mh}{0x14292967}\PY{p}{,} \PY{l+m+mh}{0x27B70A85}\PY{p}{,} \PY{l+m+mh}{0x2E1B2138}\PY{p}{,} \PY{l+m+mh}{0x4D2C6DFC}\PY{p}{,} \PY{l+m+mh}{0x53380D13}\PY{p}{,}
    \PY{l+m+mh}{0x650A7354}\PY{p}{,} \PY{l+m+mh}{0x766A0ABB}\PY{p}{,} \PY{l+m+mh}{0x81C2C92E}\PY{p}{,} \PY{l+m+mh}{0x92722C85}\PY{p}{,} \PY{l+m+mh}{0xA2BFE8A1}\PY{p}{,} \PY{l+m+mh}{0xA81A664B}\PY{p}{,}
    \PY{l+m+mh}{0xC24B8B70}\PY{p}{,} \PY{l+m+mh}{0xC76C51A3}\PY{p}{,} \PY{l+m+mh}{0xD192E819}\PY{p}{,} \PY{l+m+mh}{0xD6990624}\PY{p}{,} \PY{l+m+mh}{0xF40E3585}\PY{p}{,} \PY{l+m+mh}{0x106AA070}\PY{p}{,}
    \PY{l+m+mh}{0x19A4C116}\PY{p}{,} \PY{l+m+mh}{0x1E376C08}\PY{p}{,} \PY{l+m+mh}{0x2748774C}\PY{p}{,} \PY{l+m+mh}{0x34B0BCB5}\PY{p}{,} \PY{l+m+mh}{0x391C0CB3}\PY{p}{,} \PY{l+m+mh}{0x4ED8AA4A}\PY{p}{,}
    \PY{l+m+mh}{0x5B9CCA4F}\PY{p}{,} \PY{l+m+mh}{0x682E6FF3}\PY{p}{,} \PY{l+m+mh}{0x748F82EE}\PY{p}{,} \PY{l+m+mh}{0x78A5636F}\PY{p}{,} \PY{l+m+mh}{0x84C87814}\PY{p}{,} \PY{l+m+mh}{0x8CC70208}\PY{p}{,}
    \PY{l+m+mh}{0x90BEFFFA}\PY{p}{,} \PY{l+m+mh}{0xA4506CEB}\PY{p}{,} \PY{l+m+mh}{0xBEF9A3F7}\PY{p}{,} \PY{l+m+mh}{0xC67178F2}\PY{p}{,}
\PY{p}{]}

\PY{n}{H\PYZus{}INIT} \PY{o}{=} \PY{p}{[}
    \PY{l+m+mh}{0x6A09E667}\PY{p}{,} \PY{l+m+mh}{0xBB67AE85}\PY{p}{,} \PY{l+m+mh}{0x3C6EF372}\PY{p}{,} \PY{l+m+mh}{0xA54FF53A}\PY{p}{,}
    \PY{l+m+mh}{0x510E527F}\PY{p}{,} \PY{l+m+mh}{0x9B05688C}\PY{p}{,} \PY{l+m+mh}{0x1F83D9AB}\PY{p}{,} \PY{l+m+mh}{0x5BE0CD19}\PY{p}{,}
\PY{p}{]}
\end{Verbatim}
\end{tcolorbox}

    \subsection{Méthodes auxiliaires}\label{muxe9thodes-auxiliaires}

\begin{itemize}
\tightlist
\item
  rotr\_int(x, n) : rotation des bits de x vers la droite (rien n'est
  perdu).
\item
  ch(x, y, z) : ``sélecteur'' bit à bit (x choisit entre y et z).
\item
  maj(x, y, z) : ``vote'' bit à bit (valeur majoritaire).
\item
  sigma0\_int / sigma1\_int : combinaisons de rotations pour bien
  brasser.
\end{itemize}

Versions chaînes de bits (pour fabriquer W{[}16..63{]}) : - rotr,
shiftr, xor, sigma0, sigma1.

But : calculer à chaque tour des valeurs intermédiaires qui diffusent et
brouillent l'information du message.

    \begin{tcolorbox}[breakable, size=fbox, boxrule=1pt, pad at break*=1mm,colback=cellbackground, colframe=cellborder]
\prompt{In}{incolor}{61}{\boxspacing}
\begin{Verbatim}[commandchars=\\\{\}]
\PY{k}{def}\PY{+w}{ }\PY{n+nf}{rotr\PYZus{}int}\PY{p}{(}\PY{n}{x}\PY{p}{,} \PY{n}{n}\PY{p}{)}\PY{p}{:}
    \PY{k}{return} \PY{p}{(}\PY{p}{(}\PY{n}{x} \PY{o}{\PYZgt{}\PYZgt{}} \PY{n}{n}\PY{p}{)} \PY{o}{|} \PY{p}{(}\PY{n}{x} \PY{o}{\PYZlt{}\PYZlt{}} \PY{p}{(}\PY{l+m+mi}{32} \PY{o}{\PYZhy{}} \PY{n}{n}\PY{p}{)}\PY{p}{)}\PY{p}{)} \PY{o}{\PYZam{}} \PY{l+m+mh}{0xFFFFFFFF}


\PY{k}{def}\PY{+w}{ }\PY{n+nf}{ch}\PY{p}{(}\PY{n}{x}\PY{p}{,} \PY{n}{y}\PY{p}{,} \PY{n}{z}\PY{p}{)}\PY{p}{:}
    \PY{k}{return} \PY{p}{(}\PY{n}{x} \PY{o}{\PYZam{}} \PY{n}{y}\PY{p}{)} \PY{o}{\PYZca{}} \PY{p}{(}\PY{o}{\PYZti{}}\PY{n}{x} \PY{o}{\PYZam{}} \PY{n}{z}\PY{p}{)}


\PY{k}{def}\PY{+w}{ }\PY{n+nf}{maj}\PY{p}{(}\PY{n}{x}\PY{p}{,} \PY{n}{y}\PY{p}{,} \PY{n}{z}\PY{p}{)}\PY{p}{:}
    \PY{k}{return} \PY{p}{(}\PY{n}{x} \PY{o}{\PYZam{}} \PY{n}{y}\PY{p}{)} \PY{o}{\PYZca{}} \PY{p}{(}\PY{n}{x} \PY{o}{\PYZam{}} \PY{n}{z}\PY{p}{)} \PY{o}{\PYZca{}} \PY{p}{(}\PY{n}{y} \PY{o}{\PYZam{}} \PY{n}{z}\PY{p}{)}


\PY{k}{def}\PY{+w}{ }\PY{n+nf}{sigma0\PYZus{}int}\PY{p}{(}\PY{n}{x}\PY{p}{)}\PY{p}{:}
    \PY{k}{return} \PY{n}{rotr\PYZus{}int}\PY{p}{(}\PY{n}{x}\PY{p}{,} \PY{l+m+mi}{2}\PY{p}{)} \PY{o}{\PYZca{}} \PY{n}{rotr\PYZus{}int}\PY{p}{(}\PY{n}{x}\PY{p}{,} \PY{l+m+mi}{13}\PY{p}{)} \PY{o}{\PYZca{}} \PY{n}{rotr\PYZus{}int}\PY{p}{(}\PY{n}{x}\PY{p}{,} \PY{l+m+mi}{22}\PY{p}{)}


\PY{k}{def}\PY{+w}{ }\PY{n+nf}{sigma1\PYZus{}int}\PY{p}{(}\PY{n}{x}\PY{p}{)}\PY{p}{:}
    \PY{k}{return} \PY{n}{rotr\PYZus{}int}\PY{p}{(}\PY{n}{x}\PY{p}{,} \PY{l+m+mi}{6}\PY{p}{)} \PY{o}{\PYZca{}} \PY{n}{rotr\PYZus{}int}\PY{p}{(}\PY{n}{x}\PY{p}{,} \PY{l+m+mi}{11}\PY{p}{)} \PY{o}{\PYZca{}} \PY{n}{rotr\PYZus{}int}\PY{p}{(}\PY{n}{x}\PY{p}{,} \PY{l+m+mi}{25}\PY{p}{)}
    

\PY{k}{def}\PY{+w}{ }\PY{n+nf}{rotr}\PY{p}{(}\PY{n}{bits}\PY{p}{,} \PY{n}{nb}\PY{p}{)}\PY{p}{:}
    \PY{k}{if} \PY{o+ow}{not} \PY{n}{bits}\PY{p}{:}
        \PY{k}{return} \PY{l+s+s2}{\PYZdq{}}\PY{l+s+s2}{\PYZdq{}}
    \PY{n}{n} \PY{o}{=} \PY{n+nb}{int}\PY{p}{(}\PY{n}{bits}\PY{p}{,} \PY{l+m+mi}{2}\PY{p}{)}
    \PY{n}{mask} \PY{o}{=} \PY{p}{(}\PY{l+m+mi}{1} \PY{o}{\PYZlt{}\PYZlt{}} \PY{n+nb}{len}\PY{p}{(}\PY{n}{bits}\PY{p}{)}\PY{p}{)} \PY{o}{\PYZhy{}} \PY{l+m+mi}{1}
    \PY{n}{nb} \PY{o}{=} \PY{n}{nb} \PY{o}{\PYZpc{}} \PY{n+nb}{len}\PY{p}{(}\PY{n}{bits}\PY{p}{)}
    \PY{n}{rotated} \PY{o}{=} \PY{p}{(}\PY{n}{n} \PY{o}{\PYZgt{}\PYZgt{}} \PY{n}{nb}\PY{p}{)} \PY{o}{|} \PY{p}{(}\PY{p}{(}\PY{n}{n} \PY{o}{\PYZam{}} \PY{p}{(}\PY{p}{(}\PY{l+m+mi}{1} \PY{o}{\PYZlt{}\PYZlt{}} \PY{n}{nb}\PY{p}{)} \PY{o}{\PYZhy{}} \PY{l+m+mi}{1}\PY{p}{)}\PY{p}{)} \PY{o}{\PYZlt{}\PYZlt{}} \PY{p}{(}\PY{n+nb}{len}\PY{p}{(}\PY{n}{bits}\PY{p}{)} \PY{o}{\PYZhy{}} \PY{n}{nb}\PY{p}{)}\PY{p}{)}
    \PY{k}{return} \PY{n+nb}{format}\PY{p}{(}\PY{n}{rotated} \PY{o}{\PYZam{}} \PY{n}{mask}\PY{p}{,} \PY{l+s+sa}{f}\PY{l+s+s2}{\PYZdq{}}\PY{l+s+s2}{0}\PY{l+s+si}{\PYZob{}}\PY{n+nb}{len}\PY{p}{(}\PY{n}{bits}\PY{p}{)}\PY{l+s+si}{\PYZcb{}}\PY{l+s+s2}{b}\PY{l+s+s2}{\PYZdq{}}\PY{p}{)}


\PY{k}{def}\PY{+w}{ }\PY{n+nf}{shiftr}\PY{p}{(}\PY{n}{bits}\PY{p}{,} \PY{n}{nb}\PY{p}{)}\PY{p}{:}
    \PY{k}{if} \PY{o+ow}{not} \PY{n}{bits}\PY{p}{:}
        \PY{k}{return} \PY{l+s+s2}{\PYZdq{}}\PY{l+s+s2}{\PYZdq{}}
    \PY{k}{return} \PY{l+s+s2}{\PYZdq{}}\PY{l+s+s2}{0}\PY{l+s+s2}{\PYZdq{}} \PY{o}{*} \PY{n}{nb} \PY{o}{+} \PY{n}{bits}\PY{p}{[}\PY{p}{:}\PY{o}{\PYZhy{}}\PY{n}{nb}\PY{p}{]} \PY{k}{if} \PY{n}{nb} \PY{o}{\PYZlt{}} \PY{n+nb}{len}\PY{p}{(}\PY{n}{bits}\PY{p}{)} \PY{k}{else} \PY{l+s+s2}{\PYZdq{}}\PY{l+s+s2}{0}\PY{l+s+s2}{\PYZdq{}} \PY{o}{*} \PY{n+nb}{len}\PY{p}{(}\PY{n}{bits}\PY{p}{)}


\PY{k}{def}\PY{+w}{ }\PY{n+nf}{xor}\PY{p}{(}\PY{n}{bits1}\PY{p}{,} \PY{n}{bits2}\PY{p}{)}\PY{p}{:}
    \PY{k}{return} \PY{l+s+s2}{\PYZdq{}}\PY{l+s+s2}{\PYZdq{}}\PY{o}{.}\PY{n}{join}\PY{p}{(}\PY{l+s+s2}{\PYZdq{}}\PY{l+s+s2}{1}\PY{l+s+s2}{\PYZdq{}} \PY{k}{if} \PY{n}{bit1} \PY{o}{!=} \PY{n}{bit2} \PY{k}{else} \PY{l+s+s2}{\PYZdq{}}\PY{l+s+s2}{0}\PY{l+s+s2}{\PYZdq{}} \PY{k}{for} \PY{n}{bit1}\PY{p}{,} \PY{n}{bit2} \PY{o+ow}{in} \PY{n+nb}{zip}\PY{p}{(}\PY{n}{bits1}\PY{p}{,} \PY{n}{bits2}\PY{p}{)}\PY{p}{)}


\PY{k}{def}\PY{+w}{ }\PY{n+nf}{sigma0}\PY{p}{(}\PY{n}{bits}\PY{p}{)}\PY{p}{:}
    \PY{k}{return} \PY{n}{xor}\PY{p}{(}\PY{n}{xor}\PY{p}{(}\PY{n}{rotr}\PY{p}{(}\PY{n}{bits}\PY{p}{,} \PY{l+m+mi}{7}\PY{p}{)}\PY{p}{,} \PY{n}{rotr}\PY{p}{(}\PY{n}{bits}\PY{p}{,} \PY{l+m+mi}{18}\PY{p}{)}\PY{p}{)}\PY{p}{,} \PY{n}{shiftr}\PY{p}{(}\PY{n}{bits}\PY{p}{,} \PY{l+m+mi}{3}\PY{p}{)}\PY{p}{)}


\PY{k}{def}\PY{+w}{ }\PY{n+nf}{sigma1}\PY{p}{(}\PY{n}{bits}\PY{p}{)}\PY{p}{:}
    \PY{k}{return} \PY{n}{xor}\PY{p}{(}\PY{n}{xor}\PY{p}{(}\PY{n}{rotr}\PY{p}{(}\PY{n}{bits}\PY{p}{,} \PY{l+m+mi}{17}\PY{p}{)}\PY{p}{,} \PY{n}{rotr}\PY{p}{(}\PY{n}{bits}\PY{p}{,} \PY{l+m+mi}{19}\PY{p}{)}\PY{p}{)}\PY{p}{,} \PY{n}{shiftr}\PY{p}{(}\PY{n}{bits}\PY{p}{,} \PY{l+m+mi}{10}\PY{p}{)}\PY{p}{)}
\end{Verbatim}
\end{tcolorbox}

    \begin{tcolorbox}[breakable, size=fbox, boxrule=1pt, pad at break*=1mm,colback=cellbackground, colframe=cellborder]
\prompt{In}{incolor}{62}{\boxspacing}
\begin{Verbatim}[commandchars=\\\{\}]
\PY{k}{def}\PY{+w}{ }\PY{n+nf}{conversionBinaire}\PY{p}{(}\PY{n}{message}\PY{p}{)}\PY{p}{:}
    \PY{k}{return} \PY{l+s+s2}{\PYZdq{}}\PY{l+s+s2}{\PYZdq{}}\PY{o}{.}\PY{n}{join}\PY{p}{(}\PY{p}{[}\PY{l+s+s2}{\PYZdq{}}\PY{l+s+si}{\PYZob{}0:08b\PYZcb{}}\PY{l+s+s2}{\PYZdq{}}\PY{o}{.}\PY{n}{format}\PY{p}{(}\PY{n}{x}\PY{p}{)} \PY{k}{for} \PY{n}{x} \PY{o+ow}{in} \PY{n}{message}\PY{o}{.}\PY{n}{encode}\PY{p}{(}\PY{l+s+s2}{\PYZdq{}}\PY{l+s+s2}{utf\PYZhy{}8}\PY{l+s+s2}{\PYZdq{}}\PY{p}{)}\PY{p}{]}\PY{p}{)}


\PY{k}{def}\PY{+w}{ }\PY{n+nf}{remplissage}\PY{p}{(}\PY{n}{message}\PY{p}{)}\PY{p}{:}
    \PY{n}{bits} \PY{o}{=} \PY{n}{conversionBinaire}\PY{p}{(}\PY{n}{message}\PY{p}{)}
    \PY{n}{message\PYZus{}size} \PY{o}{=} \PY{n+nb}{format}\PY{p}{(}\PY{n+nb}{len}\PY{p}{(}\PY{n}{message}\PY{o}{.}\PY{n}{encode}\PY{p}{(}\PY{l+s+s2}{\PYZdq{}}\PY{l+s+s2}{utf\PYZhy{}8}\PY{l+s+s2}{\PYZdq{}}\PY{p}{)}\PY{p}{)} \PY{o}{*} \PY{l+m+mi}{8}\PY{p}{,} \PY{l+s+s2}{\PYZdq{}}\PY{l+s+s2}{064b}\PY{l+s+s2}{\PYZdq{}}\PY{p}{)}
    \PY{n}{bits} \PY{o}{+}\PY{o}{=} \PY{l+s+s2}{\PYZdq{}}\PY{l+s+s2}{1}\PY{l+s+s2}{\PYZdq{}}
    \PY{n}{bits} \PY{o}{+}\PY{o}{=} \PY{l+s+s2}{\PYZdq{}}\PY{l+s+s2}{0}\PY{l+s+s2}{\PYZdq{}} \PY{o}{*} \PY{p}{(}\PY{p}{(}\PY{l+m+mi}{512} \PY{o}{\PYZhy{}} \PY{l+m+mi}{64}\PY{p}{)} \PY{o}{\PYZhy{}} \PY{n+nb}{len}\PY{p}{(}\PY{n}{bits}\PY{p}{)} \PY{o}{\PYZpc{}} \PY{l+m+mi}{512}\PY{p}{)}
    \PY{n}{bits} \PY{o}{+}\PY{o}{=} \PY{n}{message\PYZus{}size}
    \PY{k}{return} \PY{n}{bits}


\PY{k}{def}\PY{+w}{ }\PY{n+nf}{décomposition}\PY{p}{(}\PY{n}{bits}\PY{p}{)}\PY{p}{:}
    \PY{k}{return} \PY{p}{[}\PY{n}{bits}\PY{p}{[}\PY{n}{i} \PY{p}{:} \PY{n}{i} \PY{o}{+} \PY{l+m+mi}{32}\PY{p}{]} \PY{k}{for} \PY{n}{i} \PY{o+ow}{in} \PY{n+nb}{range}\PY{p}{(}\PY{l+m+mi}{0}\PY{p}{,} \PY{n+nb}{len}\PY{p}{(}\PY{n}{bits}\PY{p}{)}\PY{p}{,} \PY{l+m+mi}{32}\PY{p}{)}\PY{p}{]}
\end{Verbatim}
\end{tcolorbox}

    \paragraph{\texorpdfstring{Fonction
\texttt{newMot(bits,\ t)}}{Fonction newMot(bits, t)}}\label{fonction-newmotbits-t}

\begin{itemize}
\tightlist
\item
  \textbf{Paramètres d'entrée :}

  \begin{itemize}
  \tightlist
  \item
    \texttt{bits} : Une liste de mots de 32 bits (en binaire)
    représentant les données décomposées et étendues.
  \item
    \texttt{t} : Un entier représentant l'indice du mot à générer.
  \end{itemize}
\item
  \textbf{Sortie :}

  \begin{itemize}
  \tightlist
  \item
    Retourne un mot de 32 bits (en binaire) calculé à partir des mots
    précédents et des fonctions auxiliaires \texttt{sigma0} et
    \texttt{sigma1} expliqués précédement.
  \end{itemize}
\item
  \textbf{Rôle dans SHA-256 :}

  \begin{itemize}
  \tightlist
  \item
    Génère un nouveau mot de 32 bits (W{[}t{]}) en combinant les
    résultats des fonctions \texttt{sigma0} et \texttt{sigma1}
    appliquées sur des mots spécifiques, ainsi que des additions modulo
    (2\^{}\{32\}). Ces mots étendus sont nécessaires pour les calculs
    dans les 64 itérations de l'algorithme SHA-256.
  \end{itemize}
\end{itemize}

\paragraph{\texorpdfstring{Fonction
\texttt{genererListMot(word)}}{Fonction genererListMot(word)}}\label{fonction-genererlistmotword}

\begin{itemize}
\tightlist
\item
  \textbf{Paramètres d'entrée :}

  \begin{itemize}
  \tightlist
  \item
    \texttt{word} : Une liste contenant les 16 premiers mots de 32 bits
    (en binaire) générés à partir du message.
  \end{itemize}
\item
  \textbf{Sortie :}

  \begin{itemize}
  \tightlist
  \item
    Retourne une liste étendue contenant 64 mots de 32 bits (en
    binaire).
  \end{itemize}
\item
  \textbf{Rôle dans SHA-256 :}

  \begin{itemize}
  \tightlist
  \item
    Étend la liste initiale de 16 mots à 64 mots en utilisant la
    fonction \texttt{newMot}. Ces mots étendus sont utilisés dans les 64
    itérations de l'algorithme SHA-256 pour effectuer les calculs
    nécessaires.
  \end{itemize}
\item
  \textbf{Paramètres de la fonction :}

  \begin{itemize}
  \tightlist
  \item
    \texttt{word} : Une liste de 64 mots de 32 bits (en binaire) générés
    à partir du message. Ces mots sont utilisés pour effectuer les
    calculs dans chaque itération.
  \item
    \texttt{H} : Une liste contenant les valeurs de hachage initiales
    (ou intermédiaires) utilisées pour démarrer les calculs.
  \end{itemize}
\item
  \textbf{Ce que fait la fonction :}

  \begin{itemize}
  \tightlist
  \item
    \textbf{Initialisation :} Une copie des valeurs de hachage
    \texttt{H} est créée dans \texttt{S} pour éviter de modifier
    directement \texttt{H} pendant les calculs.
  \item
    \textbf{Boucle principale (64 itérations) :}

    \begin{itemize}
    \tightlist
    \item
      À chaque itération, un mot de 32 bits (\texttt{word{[}i{]}}) est
      converti en entier (\texttt{w\_int}).
    \item
      Deux valeurs temporaires, \texttt{T1} et \texttt{T2}, sont
      calculées en utilisant des fonctions auxiliaires
      (\texttt{sigma1\_int}, \texttt{ch}, \texttt{sigma0\_int},
      \texttt{maj}) et des constantes (\texttt{K{[}i{]}}) définits
      précédement.
    \item
      Les variables temporaires \texttt{S} sont mis à jour en effectuant
      des décalages et des additions basées sur \texttt{T1} et
      \texttt{T2}.
    \end{itemize}
  \item
    \textbf{Mise à jour des valeurs de hachage :} Après les 64
    itérations, les valeurs de \texttt{S} sont ajoutées aux valeurs
    intermédiaires de \texttt{H}.
  \item
    \textbf{Retourne :} La liste \texttt{H} mise à jour, qui contient
    les nouvelles valeurs de hachage après le traitement d'un bloc de
    512 bits.
  \end{itemize}
\end{itemize}

    \begin{tcolorbox}[breakable, size=fbox, boxrule=1pt, pad at break*=1mm,colback=cellbackground, colframe=cellborder]
\prompt{In}{incolor}{63}{\boxspacing}
\begin{Verbatim}[commandchars=\\\{\}]
\PY{k}{def}\PY{+w}{ }\PY{n+nf}{newMot}\PY{p}{(}\PY{n}{bits}\PY{p}{,} \PY{n}{t}\PY{p}{)}\PY{p}{:}
    \PY{n}{s1} \PY{o}{=} \PY{n+nb}{int}\PY{p}{(}\PY{n}{sigma1}\PY{p}{(}\PY{n}{bits}\PY{p}{[}\PY{n}{t} \PY{o}{\PYZhy{}} \PY{l+m+mi}{2}\PY{p}{]}\PY{p}{)}\PY{p}{,} \PY{l+m+mi}{2}\PY{p}{)}
    \PY{n}{w7} \PY{o}{=} \PY{n+nb}{int}\PY{p}{(}\PY{n}{bits}\PY{p}{[}\PY{n}{t} \PY{o}{\PYZhy{}} \PY{l+m+mi}{7}\PY{p}{]}\PY{p}{,} \PY{l+m+mi}{2}\PY{p}{)}
    \PY{n}{s0} \PY{o}{=} \PY{n+nb}{int}\PY{p}{(}\PY{n}{sigma0}\PY{p}{(}\PY{n}{bits}\PY{p}{[}\PY{n}{t} \PY{o}{\PYZhy{}} \PY{l+m+mi}{15}\PY{p}{]}\PY{p}{)}\PY{p}{,} \PY{l+m+mi}{2}\PY{p}{)}
    \PY{n}{w16} \PY{o}{=} \PY{n+nb}{int}\PY{p}{(}\PY{n}{bits}\PY{p}{[}\PY{n}{t} \PY{o}{\PYZhy{}} \PY{l+m+mi}{16}\PY{p}{]}\PY{p}{,} \PY{l+m+mi}{2}\PY{p}{)}
    \PY{n}{result} \PY{o}{=} \PY{p}{(}\PY{n}{s1} \PY{o}{+} \PY{n}{w7} \PY{o}{+} \PY{n}{s0} \PY{o}{+} \PY{n}{w16}\PY{p}{)} \PY{o}{\PYZpc{}} \PY{p}{(}\PY{l+m+mi}{2}\PY{o}{*}\PY{o}{*}\PY{l+m+mi}{32}\PY{p}{)}
    \PY{k}{return} \PY{n+nb}{format}\PY{p}{(}\PY{n}{result}\PY{p}{,} \PY{l+s+s2}{\PYZdq{}}\PY{l+s+s2}{032b}\PY{l+s+s2}{\PYZdq{}}\PY{p}{)}


\PY{k}{def}\PY{+w}{ }\PY{n+nf}{genererListMot}\PY{p}{(}\PY{n}{word}\PY{p}{)}\PY{p}{:}
    \PY{k}{for} \PY{n}{i} \PY{o+ow}{in} \PY{n+nb}{range}\PY{p}{(}\PY{l+m+mi}{16}\PY{p}{,} \PY{l+m+mi}{64}\PY{p}{)}\PY{p}{:}
        \PY{n}{word}\PY{o}{.}\PY{n}{append}\PY{p}{(}\PY{n}{newMot}\PY{p}{(}\PY{n}{word}\PY{p}{,} \PY{n}{i}\PY{p}{)}\PY{p}{)}
    \PY{k}{return} \PY{n}{word}
\end{Verbatim}
\end{tcolorbox}

    \subsubsection{\texorpdfstring{Explication simple de
\texttt{iterateHash(word,\ H)}}{Explication simple de iterateHash(word, H)}}\label{explication-simple-de-iteratehashword-h}

\begin{itemize}
\item
  \textbf{Objectif}: traiter un bloc (64 mots) et mettre à jour l'état
  de hachage.
\item
  \textbf{Entrées}

  \begin{itemize}
  \tightlist
  \item
    \texttt{word}: 64 mots de 32 bits (W{[}0{]}..W{[}63{]}) issus du
    bloc.
  \item
    \texttt{H}: 8 nombres de 32 bits (état courant). On les copie dans
    \texttt{S} pour travailler.
  \end{itemize}
\item
  \textbf{Boucle de 64 tours (i = 0..63)}

  \begin{itemize}
  \tightlist
  \item
    On lit le mot du tour: \texttt{w\_int\ =\ int(word{[}i{]},\ 2)}.
  \item
    On calcule deux valeurs temporaires:

    \begin{itemize}
    \tightlist
    \item
      \texttt{T1\ =\ h\ +\ σ1(e)\ +\ Ch(e,f,g)\ +\ K{[}i{]}\ +\ W{[}i{]}}
      (mod 2\^{}32)
    \item
      \texttt{T2\ =\ Σ0(a)\ +\ Maj(a,b,c)} (mod 2\^{}32)
    \item
      Intuition: \texttt{T1} injecte le mot du message et la constante
      du tour; \texttt{T2} structure le mélange avec des rotations et
      une ``majorité''.
    \end{itemize}
  \item
    On met à jour les 8 registres (rotation des rôles):

    \begin{itemize}
    \tightlist
    \item
      h ← g, g ← f, f ← e
    \item
      e ← d + T1 (mod 2\^{}32)
    \item
      d ← c, c ← b, b ← a
    \item
      a ← T1 + T2 (mod 2\^{}32)
    \item
      Intuition: on ``fait tourner'' l'état et on y injecte
      \texttt{T1}/\texttt{T2} pour diffuser l'info.
    \end{itemize}
  \end{itemize}
\item
  \textbf{Rétro‑addition à la fin}

  \begin{itemize}
  \tightlist
  \item
    Après les 64 tours, on additionne \texttt{S} dans \texttt{H} mot à
    mot (mod 2\^{}32).
  \item
    Intuition: on ``cumule'' l'effet de ce bloc avec l'état global
    (feed‑forward).
  \end{itemize}
\item
  \textbf{Détails pratiques}

  \begin{itemize}
  \tightlist
  \item
    Le \texttt{\&\ 0xFFFFFFFF} force les résultats à rester sur 32 bits
    (arithmétique modulo 2\^{}32).
  \item
    Correspondance habituelle (pour se repérer): \texttt{a=S{[}0{]}},
    \texttt{b=S{[}1{]}}, \texttt{c=S{[}2{]}}, \texttt{d=S{[}3{]}},
    \texttt{e=S{[}4{]}}, \texttt{f=S{[}5{]}}, \texttt{g=S{[}6{]}},
    \texttt{h=S{[}7{]}}.
  \end{itemize}
\item
  \textbf{À retenir}

  \begin{itemize}
  \tightlist
  \item
    64 petits pas qui brouillent fortement les bits du bloc.
  \item
    \texttt{W{[}i{]}} apporte le message au tour i, \texttt{K{[}i{]}}
    ajoute de la variété fixe, \texttt{σ/Ch/Maj} assurent le bon
    mélange.
  \end{itemize}
\end{itemize}

    \begin{tcolorbox}[breakable, size=fbox, boxrule=1pt, pad at break*=1mm,colback=cellbackground, colframe=cellborder]
\prompt{In}{incolor}{64}{\boxspacing}
\begin{Verbatim}[commandchars=\\\{\}]
\PY{k}{def}\PY{+w}{ }\PY{n+nf}{iterateHash}\PY{p}{(}\PY{n}{word}\PY{p}{,} \PY{n}{H}\PY{p}{)}\PY{p}{:}
    \PY{n}{S} \PY{o}{=} \PY{n}{H}\PY{o}{.}\PY{n}{copy}\PY{p}{(}\PY{p}{)}
    \PY{k}{for} \PY{n}{i} \PY{o+ow}{in} \PY{n+nb}{range}\PY{p}{(}\PY{l+m+mi}{64}\PY{p}{)}\PY{p}{:}
        \PY{n}{w\PYZus{}int} \PY{o}{=} \PY{n+nb}{int}\PY{p}{(}\PY{n}{word}\PY{p}{[}\PY{n}{i}\PY{p}{]}\PY{p}{,} \PY{l+m+mi}{2}\PY{p}{)}
        \PY{n}{T1} \PY{o}{=} \PY{p}{(}
            \PY{n}{S}\PY{p}{[}\PY{l+m+mi}{7}\PY{p}{]} \PY{o}{+} \PY{n}{sigma1\PYZus{}int}\PY{p}{(}\PY{n}{S}\PY{p}{[}\PY{l+m+mi}{4}\PY{p}{]}\PY{p}{)} \PY{o}{+} \PY{n}{ch}\PY{p}{(}\PY{n}{S}\PY{p}{[}\PY{l+m+mi}{4}\PY{p}{]}\PY{p}{,} \PY{n}{S}\PY{p}{[}\PY{l+m+mi}{5}\PY{p}{]}\PY{p}{,} \PY{n}{S}\PY{p}{[}\PY{l+m+mi}{6}\PY{p}{]}\PY{p}{)} \PY{o}{+} \PY{n}{K}\PY{p}{[}\PY{n}{i}\PY{p}{]} \PY{o}{+} \PY{n}{w\PYZus{}int}
        \PY{p}{)} \PY{o}{\PYZam{}} \PY{l+m+mh}{0xFFFFFFFF}
        \PY{n}{T2} \PY{o}{=} \PY{p}{(}\PY{n}{sigma0\PYZus{}int}\PY{p}{(}\PY{n}{S}\PY{p}{[}\PY{l+m+mi}{0}\PY{p}{]}\PY{p}{)} \PY{o}{+} \PY{n}{maj}\PY{p}{(}\PY{n}{S}\PY{p}{[}\PY{l+m+mi}{0}\PY{p}{]}\PY{p}{,} \PY{n}{S}\PY{p}{[}\PY{l+m+mi}{1}\PY{p}{]}\PY{p}{,} \PY{n}{S}\PY{p}{[}\PY{l+m+mi}{2}\PY{p}{]}\PY{p}{)}\PY{p}{)} \PY{o}{\PYZam{}} \PY{l+m+mh}{0xFFFFFFFF}
        \PY{n}{S}\PY{p}{[}\PY{l+m+mi}{7}\PY{p}{]} \PY{o}{=} \PY{n}{S}\PY{p}{[}\PY{l+m+mi}{6}\PY{p}{]}
        \PY{n}{S}\PY{p}{[}\PY{l+m+mi}{6}\PY{p}{]} \PY{o}{=} \PY{n}{S}\PY{p}{[}\PY{l+m+mi}{5}\PY{p}{]}
        \PY{n}{S}\PY{p}{[}\PY{l+m+mi}{5}\PY{p}{]} \PY{o}{=} \PY{n}{S}\PY{p}{[}\PY{l+m+mi}{4}\PY{p}{]}
        \PY{n}{S}\PY{p}{[}\PY{l+m+mi}{4}\PY{p}{]} \PY{o}{=} \PY{p}{(}\PY{n}{S}\PY{p}{[}\PY{l+m+mi}{3}\PY{p}{]} \PY{o}{+} \PY{n}{T1}\PY{p}{)} \PY{o}{\PYZam{}} \PY{l+m+mh}{0xFFFFFFFF}
        \PY{n}{S}\PY{p}{[}\PY{l+m+mi}{3}\PY{p}{]} \PY{o}{=} \PY{n}{S}\PY{p}{[}\PY{l+m+mi}{2}\PY{p}{]}
        \PY{n}{S}\PY{p}{[}\PY{l+m+mi}{2}\PY{p}{]} \PY{o}{=} \PY{n}{S}\PY{p}{[}\PY{l+m+mi}{1}\PY{p}{]}
        \PY{n}{S}\PY{p}{[}\PY{l+m+mi}{1}\PY{p}{]} \PY{o}{=} \PY{n}{S}\PY{p}{[}\PY{l+m+mi}{0}\PY{p}{]}
        \PY{n}{S}\PY{p}{[}\PY{l+m+mi}{0}\PY{p}{]} \PY{o}{=} \PY{p}{(}\PY{n}{T1} \PY{o}{+} \PY{n}{T2}\PY{p}{)} \PY{o}{\PYZam{}} \PY{l+m+mh}{0xFFFFFFFF}
    \PY{k}{for} \PY{n}{i} \PY{o+ow}{in} \PY{n+nb}{range}\PY{p}{(}\PY{l+m+mi}{8}\PY{p}{)}\PY{p}{:}
        \PY{n}{H}\PY{p}{[}\PY{n}{i}\PY{p}{]} \PY{o}{=} \PY{p}{(}\PY{n}{H}\PY{p}{[}\PY{n}{i}\PY{p}{]} \PY{o}{+} \PY{n}{S}\PY{p}{[}\PY{n}{i}\PY{p}{]}\PY{p}{)} \PY{o}{\PYZam{}} \PY{l+m+mh}{0xFFFFFFFF}
    \PY{k}{return} \PY{n}{H}
\end{Verbatim}
\end{tcolorbox}

    \subsubsection{\texorpdfstring{Explication de la fonction
\texttt{sha256(message)}}{Explication de la fonction sha256(message)}}\label{explication-de-la-fonction-sha256message}

La fonction \texttt{sha256} implémente l'algorithme de hachage SHA-256.
Voici une explication étape par étape avec des mots simples :

\begin{enumerate}
\def\labelenumi{\arabic{enumi}.}
\tightlist
\item
  \textbf{Initialisation des valeurs de hachage :}

  \begin{itemize}
  \tightlist
  \item
    On commence par copier les valeurs initiales de hachage
    (\texttt{H\_INIT}), qui sont des constantes définies au début de
    l'algorithme.
  \end{itemize}
\item
  \textbf{Remplissage du message :}

  \begin{itemize}
  \tightlist
  \item
    Le message est converti en une chaîne binaire, puis ``rempli''
    (padding) pour que sa longueur soit un multiple de 512 bits, comme
    l'exige SHA-256.
  \end{itemize}
\item
  \textbf{Traitement par blocs de 512 bits :}

  \begin{itemize}
  \tightlist
  \item
    Le message est divisé en blocs de 512 bits.
  \item
    Chaque bloc est décomposé en mots de 32 bits.
  \item
    Des mots supplémentaires sont générés pour étendre la liste à 64
    mots.
  \item
    La fonction \texttt{iterateHash} est appelée pour mettre à jour les
    valeurs de hachage en fonction des mots et des constantes de
    l'algorithme.
  \end{itemize}
\item
  \textbf{Construction du résultat final :}

  \begin{itemize}
  \tightlist
  \item
    Une fois tous les blocs traités, les valeurs finales de hachage
    (contenues dans \texttt{H}) sont converties en une chaîne
    hexadécimale.
  \item
    Cette chaîne représente le hash final du message.
  \end{itemize}
\end{enumerate}

\subsubsection{Résumé}\label{ruxe9sumuxe9}

La fonction prend un message en entrée, le transforme en blocs de 512
bits, applique des opérations mathématiques et logiques pour chaque
bloc, et retourne un hash unique de 256 bits (64 caractères
hexadécimaux). Ce hash est une empreinte numérique du message, utilisée
pour garantir son intégrité.

    \begin{tcolorbox}[breakable, size=fbox, boxrule=1pt, pad at break*=1mm,colback=cellbackground, colframe=cellborder]
\prompt{In}{incolor}{65}{\boxspacing}
\begin{Verbatim}[commandchars=\\\{\}]
\PY{k}{def}\PY{+w}{ }\PY{n+nf}{sha256}\PY{p}{(}\PY{n}{message}\PY{p}{)}\PY{p}{:}
    \PY{n}{H} \PY{o}{=} \PY{n}{H\PYZus{}INIT}\PY{o}{.}\PY{n}{copy}\PY{p}{(}\PY{p}{)}
    \PY{n}{bin\PYZus{}str} \PY{o}{=} \PY{n}{remplissage}\PY{p}{(}\PY{n}{message}\PY{p}{)}

    \PY{k}{for} \PY{n}{i} \PY{o+ow}{in} \PY{n+nb}{range}\PY{p}{(}\PY{l+m+mi}{0}\PY{p}{,} \PY{n+nb}{len}\PY{p}{(}\PY{n}{bin\PYZus{}str}\PY{p}{)}\PY{p}{,} \PY{l+m+mi}{512}\PY{p}{)}\PY{p}{:}
        \PY{n}{block} \PY{o}{=} \PY{n}{bin\PYZus{}str}\PY{p}{[}\PY{n}{i} \PY{p}{:} \PY{n}{i} \PY{o}{+} \PY{l+m+mi}{512}\PY{p}{]}
        \PY{n}{word} \PY{o}{=} \PY{n}{genererListMot}\PY{p}{(}\PY{n}{décomposition}\PY{p}{(}\PY{n}{block}\PY{p}{)}\PY{p}{)}
        \PY{n}{H} \PY{o}{=} \PY{n}{iterateHash}\PY{p}{(}\PY{n}{word}\PY{p}{,} \PY{n}{H}\PY{p}{)}

    \PY{n}{result} \PY{o}{=} \PY{l+s+s2}{\PYZdq{}}\PY{l+s+s2}{\PYZdq{}}
    \PY{k}{for} \PY{n}{h} \PY{o+ow}{in} \PY{n}{H}\PY{p}{:}
        \PY{n}{result} \PY{o}{+}\PY{o}{=} \PY{n+nb}{format}\PY{p}{(}\PY{n}{h}\PY{p}{,} \PY{l+s+s2}{\PYZdq{}}\PY{l+s+s2}{08x}\PY{l+s+s2}{\PYZdq{}}\PY{p}{)}
    \PY{k}{return} \PY{n}{result}
\end{Verbatim}
\end{tcolorbox}

    \begin{tcolorbox}[breakable, size=fbox, boxrule=1pt, pad at break*=1mm,colback=cellbackground, colframe=cellborder]
\prompt{In}{incolor}{66}{\boxspacing}
\begin{Verbatim}[commandchars=\\\{\}]
\PY{c+c1}{\PYZsh{} Exemple d\PYZsq{}utilisation}
\PY{n}{message} \PY{o}{=} \PY{l+s+s2}{\PYZdq{}}\PY{l+s+s2}{abc}\PY{l+s+s2}{\PYZdq{}}
\PY{n}{result} \PY{o}{=} \PY{n}{sha256}\PY{p}{(}\PY{n}{message}\PY{p}{)}
\PY{n+nb}{print}\PY{p}{(}\PY{l+s+sa}{f}\PY{l+s+s2}{\PYZdq{}}\PY{l+s+s2}{SHA\PYZhy{}256 de }\PY{l+s+s2}{\PYZsq{}}\PY{l+s+si}{\PYZob{}}\PY{n}{message}\PY{l+s+si}{\PYZcb{}}\PY{l+s+s2}{\PYZsq{}}\PY{l+s+s2}{: }\PY{l+s+si}{\PYZob{}}\PY{n}{result}\PY{l+s+si}{\PYZcb{}}\PY{l+s+s2}{\PYZdq{}}\PY{p}{)}
\end{Verbatim}
\end{tcolorbox}

    \begin{Verbatim}[commandchars=\\\{\}]
SHA-256 de 'abc':
ba7816bf8f01cfea414140de5dae2223b00361a396177a9cb410ff61f20015ad
    \end{Verbatim}


    % Add a bibliography block to the postdoc
    
    
    
\end{document}
